\documentclass{oxcsletter}
\usepackage{url}
\signature{Gary Mirams, on behalf of,\\Christopher Arthurs, Miguel Bernabeu, Rafael Bordas, Jonathan Cooper, Alberto Corrias, Yohan Davit, Sara-Jane Dunn, Alexander Fletcher, Daniel Harvey, James Osborne, Pras Pathmanathan, Joe Pitt-Francis, James Southern, Nejib Zemzemi, David Gavaghan}
\phone{610671}
\fax{273839}
\econtact{gary.mirams}

\begin{document}

\begin{letter}{}

\subject{Pre-submission enquiry to PLoS Computational Biology}  

\opening{Dear Editors,}

I am writing a pre-submission enquiry about a possible open-source computational biology software article (as per \url{www.ploscompbiol.org/static/guidelines.action#software}).
The Computational Biology Group here in Oxford leads the development of `\textbf{Chaste}': \textbf{C}ancer, \textbf{H}eart \textbf{A}nd \textbf{S}oft \textbf{T}issue \textbf{E}nvironment. 
We would like to contribute an article to PLoS Computational Biology describing this software. 
Chaste development began in 2005, and the latest version 3.0 was released in December 2011.
All code, including the latest development version, is freely available to download via \url{www.cs.ox.ac.uk/chaste} under a GNU LGPL licence (v2.1).

The software is based on C++ and a number of third party libraries (all open source), it provides a framework of commonly used code for Computational Biology problems. 
The software is comprised of a number of modules common to many simulation applications (meshes, linear systems, ODE/PDE solvers, continuum mechanics, input/output etc.) together with more specialised modules for individual applications --- in particular for cardiac electro-physiology/mechanics and for cell-based tissue/cancer modelling. 
Chaste recently performed very well in an accuracy benchmark of all major cardiac electrophysiology codes, and is the only one that is open-source; 
furthermore it has been optimised for use on parallel high performance computing facilities, also making it one of the fastest for this application.

We have a public wiki with installation instructions and tutorials (\url{https://chaste.cs.ox.ac.uk/cgi-bin/trac.cgi/wiki/}). 
A community of users from around the world has developed, including the US Food and Drug Administration (to study drug-induced cardiotoxicity) and NASA (to study radiation effects in tissue). 
We support an active users' mailing list which allows us to remotely assist in the use of Chaste and its extension for a variety of applications. 
As such, the software is being used for diverse applications, such as pattern formation, developmental biology, bacterial biofilms, lung mechanics, sexually transmitted infections and immunology. 

To date over 15 publications have been written on aspects of the computer science/numerical analysis undertaken as part of Chaste software and algorithm development. 
These software papers have been cited over 20 times by members of the wider community studying a variety of applications.
A similar number of publications have been written by our team on novel biological/physiological applications using Chaste, such as defibrillation of the human heart and the mechanisms behind the initiation of colorectal cancer.

We would like to contribute an article to PLoS Computational Biology describing Chaste, together with its current and potential uses, to a wide audience of researchers in our field.

\closing{Yours sincerely}
\end{letter}
\end{document}
