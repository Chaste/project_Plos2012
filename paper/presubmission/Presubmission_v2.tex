\documentclass{oxcsletter}
\usepackage{url}
\signature{Gary Mirams, on behalf of,\\Christopher Arthurs, Miguel Bernabeu, Rafael Bordas, Jonathan Cooper, Alberto Corrias, Yohan Davit, Sara-Jane Dunn, Alexander Fletcher, Daniel Harvey, James Osborne, Pras Pathmanathan, Joe Pitt-Francis, James Southern, Nejib Zemzemi, David Gavaghan}
\phone{610671}
\fax{273839}
\econtact{gary.mirams}

\begin{document}

\begin{letter}{}

\subject{Pre-submission enquiry to PLoS Computational Biology}  

\opening{Dear Editors,}

Thank you for your reply to our pre-submission enquiry.
We are making another enquiry, as we believe we did not explain adequately the significant advances that have taken place in the Chaste software since the first release.
As you noted, the only article to describe the Chaste package as a whole was the Computer Physics Communications article of 2009, which described Chaste v1.0 released in March 2009.
Since then a large team has been working on the code for one or two days each week.
There have now been six releases, with the current version 3.0 having been released in December 2011.
The code-base is roughly three times the size it was when previously described, has been adopted by many different groups, 
and is employed on a far wider variety of applications, some of which we touched upon in our previous letter.

We would like to publish a software article in PLoS Computational Biology summarising some of the large advances we have made since releasing v1.0 in March 2009, 
showing some of the applications that Chaste can now be utilized to perform.

Just some of the improvements include:
\begin{itemize}
 \item \textbf{Global}
 \begin{itemize}
  \item A distributed tetrahedral mesh is now supported, allowing large meshes (of tens of millions of nodes and elements) to be loaded on High Performance Computing clusters, and for this to be partitioned efficiently using parMETIS.
  \item Support for multiple versions of library dependencies (particularly PETSc and Boost) and development of an Ubuntu package have improved the ease of installation and maintenance.
  \item Visualization via VTK (Paraview) and Cmgui in addition to Meshalyzer.
  \item Improved and new tutorials and user guides.
  \item Public access to the latest revision of the code and tools for external developers.
 \end{itemize}

 \item \textbf{Cell-based code}
 \begin{itemize}
  \item Version 1.0 supported 1--3D tetrahedral-mesh-based 'spring and cell-centre' models only. 
Chaste now allows the description of cell-based simulations in multiple modelling frameworks (and is unique in allowing their comparison):
\begin{itemize}
 \item 1--3D node/overlapping-sphere style models,
 \item 2/3D cellular Potts models,
 \item 2D vertex-based cell simulations.
\end{itemize}
 \item Support for generic systems of cellular-ODEs with coupled reaction-diffusion PDEs. These models are used in many areas of mathematical biology including tissue development and pattern formation.

 \end{itemize}
 \item \textbf{Cardiac code}
 \begin{itemize}
  \item The electrophysiology solvers in Chaste v1.0 scaled to 64 cores at 50\% efficiency (Figure 7 of the CPC 2009 paper), v3.0 now scales up to 2048 cores at 73\% efficiency.
  Through additional improvements to numerical algorithms the code is therefore capable of performing electrophysiology simulations thousands of times faster than in 2009.
  \item Serialization code allows checkpointing of long simulations. 
  Code can be saved and reloaded on different numbers of processors and different architectures.
  \item Support for bidomain simulations with a `bath', for simulation of an electrocardiogram.
  \item Solvers for electro-mechanical simulations have been introduced.
  \item CellML files can now be dynamically loaded `on-the-fly'.
  \item Multiple improvements to CellML handling allow automatic units conversion, optimised models for single-cell simulations, and parameter and derived quantity handling.
  \item Support for the modelling of drug action on cardiac tissue.
  \item Improvments in both the diversity and performance of post-processing.
 \end{itemize}
\end{itemize}

Note that these improvements have enabled a wider range of applications than those previously described. 
Many of which have generated exceptional biological insight in a variety of fields (some of these are mentioned in our previous correspondence, attached below).
Since our previous correspondence we are now also making the code available under the BSD 3 clause licence to improve access and re-usability for our industrial collaborators. 

I hope this makes clear how much extra Chaste has to offer since its initial description three years ago, 
and that you would like to allow us to describe the current capabilities of Chaste to the readers of PLoS Computational Biology.

\closing{Yours sincerely}
\newpage
\textbf{17th Feb: Our original pre-submission enquiry}\\
I am writing a pre-submission enquiry about a possible open-source computational biology software article (as per \url{www.ploscompbiol.org/static/guidelines.action#software}).
The Computational Biology Group here in Oxford leads the development of `\textbf{Chaste}': \textbf{C}ancer, \textbf{H}eart \textbf{A}nd \textbf{S}oft \textbf{T}issue \textbf{E}nvironment. 
We would like to contribute an article to PLoS Computational Biology describing this software. 
Chaste development began in 2005, and the latest version 3.0 was released in December 2011.
All code, including the latest development version, is freely available to download via \url{www.cs.ox.ac.uk/chaste} under a GNU LGPL licence (v2.1).

The software is based on C++ and a number of third party libraries (all open source), it provides a framework of commonly used code for Computational Biology problems. 
The software is comprised of a number of modules common to many simulation applications (meshes, linear systems, ODE/PDE solvers, continuum mechanics, input/output etc.) together with more specialised modules for individual applications --- in particular for cardiac electro-physiology/mechanics and for cell-based tissue/cancer modelling. 
Chaste recently performed very well in an accuracy benchmark of all major cardiac electrophysiology codes, and is the only one that is open-source; 
furthermore it has been optimised for use on parallel high performance computing facilities, also making it one of the fastest for this application.

We have a public wiki with installation instructions and tutorials (\url{https://chaste.cs.ox.ac.uk/cgi-bin/trac.cgi/wiki/}). 
A community of users from around the world has developed, including the US Food and Drug Administration (to study drug-induced cardiotoxicity) and NASA (to study radiation effects in tissue). 
We support an active users' mailing list which allows us to remotely assist in the use of Chaste and its extension for a variety of applications. 
As such, the software is being used for diverse applications, such as pattern formation, developmental biology, bacterial biofilms, lung mechanics, sexually transmitted infections and immunology. 

To date over 15 publications have been written on aspects of the computer science/numerical analysis undertaken as part of Chaste software and algorithm development. 
These software papers have been cited over 20 times by members of the wider community studying a variety of applications.
A similar number of publications have been written by our team on novel biological/physiological applications using Chaste, such as defibrillation of the human heart and the mechanisms behind the initiation of colorectal cancer.

We would like to contribute an article to PLoS Computational Biology describing Chaste, together with its current and potential uses, to a wide audience of researchers in our field.

\textbf{23rd Feb: Software editor's reply:}\\
Dear Dr. Mirams,

Thanks for your presubmission inquiry regarding your manuscript 'Chaste: an open-source software for computational biology problems'.  
While your software meets some of our criteria for software articles (http://www.ploscompbiol.org/static/guidelines.action\#software ) the current abstract does not. 
Since there already is a detailed article about Chaste (Comp. Phys. Comm. 2009) a manuscript needs to demonstrate significant enough advance for us to consider publishing in PLoS Computational Biology. 
The current presubmission inquiry is not clear in this regard. 
A manuscript with a different focus might however be suitable.

We very much appreciate your wish to present your work in an open-access publication and therefore want to alert you to an alternative that you may find attractive. 
PLoS ONE is a unique swift, high-volume system for the publication of peer-reviewed research from any scientific discipline.  
PLoS ONE aims to exploit the full potential of the web to make the most of every piece of research; 
if you would like to submit your work to PLoS ONE, please visit www.plosone.org and submit your work online.

Thanks for considering PLoS Computational Biology, and good luck with your work.

Yours sincerely,

Andreas Prlic
Software Editor
\end{letter}
\end{document}
